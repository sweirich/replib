% \documentclass{article}

% \usepackage{hcar}
% \usepackage{graphicx}
% \usepackage{paralist}

% \begin{document}

\begin{hcarentry}[new]{Unbound}
\report{Brent Yorgey}
\status{active development}
\participants{Stephanie Weirich, Tim Sheard}
\makeheader

Unbound is a new domain-specific language and library for working with
binding structure.  Implemented on top of the RepLib generic
programming framework, it automatically provides operations such as
alpha equivalence, capture-avoiding substitution, and free variable
calculation for user-defined data types, requiring only a tiny bit of
boilerplate on the part of the user.  It features a simple yet rich
combinator language for binding specifications, including support for
pattern binding, type annotations, recursive binding, nested binding,
and multiple atom types.

Work is ongoing to extend Unbound with support for generalized
abstract data types (GADTs) and other features.

\FurtherReading
\begin{compactitem}
\item \url{http://byorgey.wordpress.com/2011/03/28/binders-unbound/}
\item \url{http://hackage.haskell.org/package/unbound}
\item \url{http://code.google.com/p/replib/}
\end{compactitem}
\end{hcarentry}

% \end{document}